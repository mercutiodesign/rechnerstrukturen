\documentclass[12pt,a4paper]{article}
\usepackage[utf8x]{inputenc}
\usepackage{ucs}
\usepackage[german]{babel}
\usepackage{amsmath}
\usepackage{amsfonts}
\usepackage{amssymb}
\usepackage{fancyhdr}
\usepackage{icomma}
\usepackage{booktabs}

\pagestyle{fancy}
\fancyhead{}
\fancyfoot{}
\fancyhead[L]{Übung Rechnerstrukturen}
\fancyhead[R]{11. November 2011}
\fancyfoot[R]{Elena Noll, Martin Dreher }
\fancyfoot[C]{\thepage}
\fancyfoot[L]{Matrikelnr: 6335415, 6354762}

\newcommand\T{\rule{0pt}{2.6ex}}
\setlength{\parindent}{0pt}

\author{Martin Dreher}
\title{Übung Rechnerstrukturen}

\begin{document}
 
 \section*{Aufgabenblatt 2}
 
 \begin{tabular}{l l l}
 \textbf{Namen: } & Elena Noll, & Martin Dreher \\
 \textbf{Matrikel: } & 6335415, & 6354762
 \end{tabular}
 
 \subsection*{Aufgabe 2.1}
 In dieser Aufgabe gehen wir von den SI-Binärpräfixen (1 KB = 1000 B) aus.
 
 \subsubsection*{(a)}
 Das Register kann $2^{64} = 18\,446\,744\,073\,709\,551\,616$ verschieden Werte annehmen. Die Taktfrequenz entspricht
 $$3,1 \textrm{ GHz} = 3\,100\,000\,000 \textrm{ Berechnungen / Sekunde}$$
 
 Somit gilt für den Überlaufzeitpunkt $t$:
 $$3\,100\,000\,000 \cdot t = 18\,446\,744\,073\,709\,551\,617 \Leftrightarrow t \approx 5\,950\,562\,604,4 \textrm{ Sekunden}$$
 
 Also läuft das Register erst nach ca. 188 Jahren über (Genauer: 188 Jahren 8 Monaten 8 Tagen 22 Stunden 3 Minuten und 24.4 Sekunden). Wenn die CPU vom 11.11.2011 an durchgehend laufen würde, würde sie am 5. Juni 2200 überlaufen.
 
 \subsubsection*{(b)}
 
 Der Überlaufzeitpunkt des kleineren Registers $t'$ ist:
 $$3\,100\,000\,000 \cdot t' = 2^{32}+1 = 4\,294\,967\,297 \Leftrightarrow t' \approx 1,385 \textrm{ Sekunden}$$
 
 Dieser Prozessor würde also schon innerhalb der zweiten Sekunde das Register durchlaufen.
 
 %%%%%%%%%%%%%%%%%%%%%%%%%%%%%%%%%%%%%%%%%%%%%%%%%%%%%%%%%%%%%%%%%%%%%%%%%%%
 %%%%%%%%%%%%%%%%%%%%%%%%%%%%%%%%%%%%%%%%%%%%%%%%%%%%%%%%%%%%%%%%%%%%%%%%%%%
 %%%%%%%%%%%%%%%%%%%%%%%%%%%%%%%%%%%%%%%%%%%%%%%%%%%%%%%%%%%%%%%%%%%%%%%%%%%
 %%%%%%%%%%%%%%%%%%%%%%%%%%%%%%%%%%%%%%%%%%%%%%%%%%%%%%%%%%%%%%%%%%%%%%%%%%%
 %%%%%%%%%%%%%%%%%%%%%%%%%%%%%%%%%%%%%%%%%%%%%%%%%%%%%%%%%%%%%%%%%%%%%%%%%%%
 %%%%%%%%%%%%%%%%%%%%%%%%%%%%%%%%%%%%%%%%%%%%%%%%%%%%%%%%%%%%%%%%%%%%%%%%%%%
 %%%%%%%%%%%%%%%%%%%%%%%%%%%%%%%%%%%%%%%%%%%%%%%%%%%%%%%%%%%%%%%%%%%%%%%%%%%
 
 


 %%%%%%%%%%%%%%%%%%%%%%%%%%%%%%%%%%%%%%%%%%%%%%%%%%%%%%%%%%%%%%%%%%%%%%%%%%%
 %%%%%%%%%%%%%%%%%%%%%%%%%%%%%%%%%%%%%%%%%%%%%%%%%%%%%%%%%%%%%%%%%%%%%%%%%%%
 %%%%%%%%%%%%%%%%%%%%%%%%%%%%%%%%%%%%%%%%%%%%%%%%%%%%%%%%%%%%%%%%%%%%%%%%%%%
 %%%%%%%%%%%%%%%%%%%%%%%%%%%%%%%%%%%%%%%%%%%%%%%%%%%%%%%%%%%%%%%%%%%%%%%%%%%
 %%%%%%%%%%%%%%%%%%%%%%%%%%%%%%%%%%%%%%%%%%%%%%%%%%%%%%%%%%%%%%%%%%%%%%%%%%%
 %%%%%%%%%%%%%%%%%%%%%%%%%%%%%%%%%%%%%%%%%%%%%%%%%%%%%%%%%%%%%%%%%%%%%%%%%%%
 
 
 
 \subsection*{2.6} 
 Das Komplement einer Zahl $z$ zur Basis $n$ lautet: 
 $$K_b(z) = \begin{cases} b^n− z & \textrm{für } z \neq 0\\
0& \textrm{für } z = 0\end{cases}$$
$$K_{b-1}(z) =  \begin{cases} b^n - b^{-m}− z & \textrm{für } z \neq 0\\
0& \textrm{für } z = 0\end{cases}$$
 Wobei $n$ die Anzahl der Vorkommastellen und $m$ die Anzahl der Nachkommastellen ist.
 \subsubsection*{(a)} Sei $n = 1$:
  
 $K_{10}(4,381)_{10} = 10^1 - 4,381 = 5,619_{10}$
 \subsubsection*{(b)}Sei $n = 0$, $m=4$:
 
 $K_9(0,4172)_{10} = 10^0 - 10^{-4} - 0,4172 = 1 - 0,0001 - 0,4172  = 0,5827_{10}$
 \subsubsection*{(c)}
 Wir rechnen nun im Dualsystem. Sei $n = 1$:
 $$K_2(1,011)_2 = 10^1 - 1,011 = 0,101_2$$
 Rechnung:
 $$
 \begin{array}{r r}
 &10,000\\
 -&\underset{1}{\;}\underset{1}{1},\underset{1}{0}\underset{1}{1}\underset{\;}{1}\\
 \hline
 &00,101 
 \end{array}$$
 \subsubsection*{(d)}Sei $n=100_2$ und $m=10_2$: 
 $$K_1(110,01)_2 = 10^{100} - 10^{-10} - 110,01 = 1000 - 0,01 - 110,01 = 1,1_2$$
 
 Rechnung:
 $$\begin{array}{r r}
 & 1000,00\\
 -&   0,01\\
 -&\underset{1}{\;}\underset{1}{1}\underset{1}{1}\underset{1}{0},\underset{1}{0}1\\
 \hline
 &0001,10
 \end{array}$$
 
 \subsection*{2.7} 
 \begin{center}

 \begin{tabular}{r||r|r|r|r|r}
 Teil & Ganzzahl & Betrag+VZ & Exzess-127 & Einerkomp. & Zweierkomp.\\
 \hline \T
 \textbf{(a) 00001011} &  11 &   11&  116&   11&   11\\
 \textbf{(b) 01100110} & 102 &  102&   15&  102&  102\\
 \textbf{(c) 10000001} & 129 &   -1&   -2& -126& -127\\
 \textbf{(d) 11111110} & 254 & -126& -127&   -1&   -2\\
 \end{tabular}
\end{center}






\end{document}