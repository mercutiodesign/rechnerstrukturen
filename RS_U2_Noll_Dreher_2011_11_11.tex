\documentclass[12pt,a4paper]{article}
\usepackage[utf8x]{inputenc}
\usepackage{ucs}
\usepackage[german]{babel}
\usepackage{amsmath}
\usepackage{amsfonts}
\usepackage{amssymb}
\usepackage{fancyhdr}
\usepackage{icomma}
\usepackage{booktabs}

\pagestyle{fancy}
\fancyhead{}
\fancyfoot{}
\fancyhead[L]{Übung Rechnerstrukturen}
\fancyhead[R]{11. November 2011}
\fancyfoot[R]{Elena Noll, Martin Dreher }
\fancyfoot[C]{\thepage}
\fancyfoot[L]{Matrikelnr: 6335415, 6354762}

\newcommand\T{\rule{0pt}{2.6ex}}
\setlength{\parindent}{0pt}

\author{Martin Dreher}
\title{Übung Rechnerstrukturen}

\begin{document}
 
 \section*{Aufgabenblatt 2}
 
 \begin{tabular}{l l l}
 \textbf{Namen: } & Elena Noll, & Martin Dreher \\
 \textbf{Matrikel: } & 6335415, & 6354762
 \end{tabular}
 
 \subsection*{Aufgabe 2.1}
 In dieser Aufgabe gehen wir von den SI-Binärpräfixen (1 KB = 1000 B) aus.
 
 \subsubsection*{(a)}
 Das Register kann $2^{64} = 18\,446\,744\,073\,709\,551\,616$ verschieden Werte annehmen. Die Taktfrequenz entspricht
 $$3,1 \textrm{ GHz} = 3\,100\,000\,000 \textrm{ Berechnungen / Sekunde}$$
 
 Somit gilt für den Überlaufzeitpunkt $t$:
 $$3\,100\,000\,000 \cdot t = 18\,446\,744\,073\,709\,551\,617 \Leftrightarrow t \approx 5\,950\,562\,604,4 \textrm{ Sekunden}$$
 
 Also läuft das Register erst nach ca. 188 Jahren über (Genauer: 188 Jahren 8 Monaten 8 Tagen 22 Stunden 3 Minuten und 24.4 Sekunden). Wenn die CPU vom 11.11.2011 an durchgehend laufen würde, würde sie am 5. Juni 2200 überlaufen.
 
 \subsubsection*{(b)}
 
 Der Überlaufzeitpunkt des kleineren Registers $t'$ ist:
 $$3\,100\,000\,000 \cdot t' = 2^{32}+1 = 4\,294\,967\,297 \Leftrightarrow t' \approx 1,385 \textrm{ Sekunden}$$
 
 Dieser Prozessor würde also schon innerhalb der zweiten Sekunde das Register durchlaufen.
 

\subsection*{Aufgabe 2.1}

\subsection*{Aufgabe 2.2}
\subsection*{a)}
\begin{align*}
49 : 2 = 24,5 \Rightarrow 1 \\
24 : 2 = 12 \Rightarrow 0 \\
12 : 2 = 6 \Rightarrow 0 \\
6 : 2 = 3 \Rightarrow 0 \\
3 : 2 = 1,5 \Rightarrow 1 \\
1 : 2 = 0,5 \Rightarrow 1 \\
\Rightarrow [49]_{10} = [110001]_{2} \\
\\
49 : 8 = 6 \Rightarrow \textrm{ \textrm{ Rest }: } 1 \\
6 : 8 = 0 \Rightarrow \textrm{ \textrm{ Rest }: } 6 \\
\Rightarrow [49]_{10} = [61]_{8} \\
\\
49 : 16 = 3 \Rightarrow \textrm{ \textrm{ Rest }: } 1 \\
3 : 16 = 0 \Rightarrow \textrm{ \textrm{ Rest }: } 3 \\
\Rightarrow [49]_{10} = [31]_{16}
\end{align*}
\subsubsection*{b)}
\begin{align*}
2011 : 2 = 1005,5 \Rightarrow 1 \\
1005 : 2 = 502,5 \Rightarrow 1 \\
502 : 2 = 251 \Rightarrow 0 \\
251 : 2 = 125,5 \Rightarrow 1 \\
125 : 2 = 62,5 \Rightarrow 1 \\
62 : 2 = 31 \Rightarrow 0 \\
31 : 2 = 15,5 \Rightarrow 1 \\
15 : 2 = 7,5 \Rightarrow 1 \\
7 : 2 = 3,5 \Rightarrow 1 \\
3 : 2 = 1,5 \Rightarrow 1 \\
1 : 2 = 0,5 \Rightarrow 1 \\
\Rightarrow [2011]_{10} = [11111011011]_{2} \\
\\
2011 : 8 = 251 \Rightarrow \textrm{ \textrm{ Rest }: } 3 \\
251 : 8 = 31 \Rightarrow \textrm{ \textrm{ Rest }: } 3 \\
31 : 8 = 3 \Rightarrow \textrm{ \textrm{ Rest }: } 7 \\
3 : 8 = 0 \Rightarrow \textrm{ \textrm{ Rest }: } 3 \\
\Rightarrow [2011]_{10} = [3733]_{8} \\
\\
2011 : 16 = 125 \Rightarrow \textrm{ \textrm{ Rest }: } 11 \\
125 : 16 = 7 \Rightarrow \textrm{ Rest } 13 \\
7 : 16 = 0 \Rightarrow \textrm{ Rest } 7 \\
\Rightarrow [2011]_{10} = [7DB]_{16}
\end{align*}
\subsubsection*{c)}
\begin{align*}
2 * 0,53125 = 1,0625 \Rightarrow \textrm{ Ziffer: } 1 \\
2 * 0,0625 = 0,125 \Rightarrow \textrm{ Ziffer: } 0 \\
2 * 0,125 = 0,25 \Rightarrow \textrm{ Ziffer: } 0 \\
2 * 0,25 = 0,5 \Rightarrow \textrm{ Ziffer: } 0 \\
2 * 0,5 = 1 \Rightarrow \textrm{ Ziffer: } 1 \\
\Rightarrow [0,53125]_{10} = [0,10001]_{2} \\
\\
8 * 0,53125 = 4,25 \Rightarrow \textrm{ Ziffer: } 4 \\
8 * 0,25 = 2 \Rightarrow \textrm{ Ziffer: } 2 \\
\Rightarrow [0,53125]_{10} = [0,42]_{8} \\
\\
16 * 0,53125 = 8,5 \Rightarrow \textrm{ Ziffer: } 8 \\
16 * 0,5 = 8 \Rightarrow \textrm{ Ziffer: } 8 \\
\Rightarrow [0,53125]_{10} = [0,88]_{16}
\end{align*}
\subsubsection*{d)}
\begin{align*}
135 : 2 = 67,5 \Rightarrow 1 \\
67 : 2 = 33,5 \Rightarrow 1 \\
33 : 2 = 16,5 \Rightarrow 1 \\
16 : 2 = 8 \Rightarrow 0 \\
8 : 2 = 4 \Rightarrow 0 \\
4 : 2 = 2 \Rightarrow 0 \\
2 : 2 = 1 \Rightarrow 0 \\
1 : 2 = 0,5 \Rightarrow 1 \\
\\
2 * 0,375 = 0,75 \Rightarrow \textrm{ Ziffer: } 0 \\
2 * 0,75 = 1,5 \Rightarrow \textrm{ Ziffer: } 1 \\
2 * 0,5 = 1 \Rightarrow \textrm{ Ziffer: } 1 \\
\Rightarrow [135,375]_{10} = [10000111,011]_{2} \\
\\
135 : 8 = 16 \Rightarrow \textrm{ \textrm{ Rest }: } 7 \\
16 : 8 = 2 \Rightarrow \textrm{ \textrm{ Rest }: } 0 \\
2 : 8 = 0 \Rightarrow \textrm{ \textrm{ Rest }: } 2 \\
\\
0,375 * 8 = 3 \Rightarrow \textrm{ Ziffer: } 3 \\
\Rightarrow [135,375]_{10} = [207,3]_{8} \\
\\
135 : 16 = 8 \Rightarrow \textrm{ \textrm{ Rest }: } 7 \\
8 : 16 = 0 \Rightarrow \textrm{ \textrm{ Rest }: } 8 \\
\\
16 * 0,375 = 6 \Rightarrow \textrm{ Ziffer: } 6 \\
\Rightarrow [135,375]_{10} = [87,6]_{16}
\end{align*}

\subsection*{2,3}
\subsubsection*{a)}
\begin{align*}
0 * 1 = 0 \\
1 * 2 = 2 \\
0 * 4 = 0 \\
1 * 8 = 8 \\
\Rightarrow 0 + 2 + 0 + 8 = 10 \\
\\
1 * 1 = 1 \\
1 : 2 = 0,5 \\
\\
\Rightarrow 10 + 0,5 = 10,5 \\
\Rightarrow [1010,1]_{2} = [10,5]_{10}
\end{align*}
\subsubsection*{b)}
\begin{align*}
0 * 1 = 0 \\
1 * 2 = 2 \\
0 * 4 = 0 \\
1 * 8 = 8 \\
1 * 16 = 16 \\
\Rightarrow 0 + 2 + 0 + 8 + 16 = 26 \\
\\
1 * 1 = 1 \\
1 * 2 = 2 \\
0 * 4 = 0 \\
0 * 8 = 0 \\
1 * 16 = 16 \\
\Rightarrow 1 + 2 + 0 + 0 + 16 = 19 \\
19 : 32 = 0,59375\\
\\
\Rightarrow 26 + 0,59375 = 26,59375 \\
\Rightarrow [11010,10011]_{2} = [26,59375]_{10}
\end{align*}

\subsection*{Aufgabe 2.4}
\begin{align*}
27355 + 16195 = 43550 \\
\\
[27355]_{10} = [110101011011011]_{2} \textrm{ (Rechnung siehe Aufgabe 2.2)} \\
[16195]_{10} = [11111101000011]_{2} \textrm{ (Rechnung siehe Aufgabe 2.2)} \\
\\
1 + 1 = 0 ~(1 \textrm{ Übertrag}) \\
1 + 1 + 1 = 1 ~(1 \textrm{ Übertrag}) \\
0 + 0 + 1 = 1 \textrm{ (kein Übertrag) } \\
1 + 0 = 1 \textrm{ (kein Übertrag) } \\
1 + 0 = 1 \textrm{ (kein Übertrag) } \\
0 + 0 = 0 \textrm{ (kein Übertrag) } \\
1 + 1 = 0 ~(1 \textrm{ Übertrag}) \\
1 + 0 + 1 = ~0 ~(1 \textrm{ Übertrag}) \\
0 + 1 + 1 = ~0 ~ (1 \textrm{ Übertrag}) \\
1 + 1 + 1 = ~1 ~(1 \textrm{ Übertrag}) \\
0 + 1 + 1 = ~0~ (1 \textrm{ Übertrag}) \\
1 + 1 + 1 = ~1~ (1 \textrm{ Übertrag}) \\
0 + 1 + 1 = ~0~ (1 \textrm{ Übertrag}) \\
1 + 1 + 1 = ~1~ (1 \textrm{ Übertrag}) \\
1 + 1 = 0 ~(1 \textrm{ Übertrag}) \\
1 = 1 \\
\Rightarrow [110101011011011]_{2} + [11111101000011]_{2} = [1010101000011110]_{2} \\
\\
[1010101000011110]_{2} = [43550]_{10} \textrm{(Rechnung siehe Aufgabe 2.3)}
\end{align*}

\subsection*{Aufgabe 2.5}
\begin{align*}
1 * 10111001 = 10111001 \\
1 * 10111001 = 10111001 \\
0 * 10111001 = 00000000 \\
1 * 10111001 = 10111001 \\
0 * 10111001 = 00000000 \\
1 * 10111001 = 10111001 \\
\\
1 = 1 ~(0 \textrm{ Übertrag}) \\
0 + 0 = 0 ~(0 \textrm{ Übertrag}) \\
0 + 0 + 1 = 1 ~(0 \textrm{ Übertrag}) \\
1 + 0 + 0 + 0 = 1 ~(0 \textrm{ Übertrag}) \\
1 + 0 + 0 + 0 + 1 = 0 ~(1 \textrm{ Übertrag}) \\
1 + 0 + 1 + 0 + 0 + 1 + 1 = 0 ~(10 \textrm{ Übertrag}) \\
0 + 0 + 1 + 0 + 0 + 0 + 0 = 1 ~(0 \textrm{ Übertrag}) \\
1 + 0 + 1 + 0 + 1 + 0 + 1 = 0 ~(10 \textrm{ Übertrag}) \\
0 + 0 + 0 + 1 + 1 + 0 = 0 ~(1 \textrm{ Übertrag}) \\
1 + 0 + 1 + 1 + 1 + 1 = 1 ~(10 \textrm{ Übertrag}) \\
0 + 0 + 1 + 0 = 1 ~(0 \textrm{ Übertrag}) \\
1 + 0 + 1 = 0 ~(1 \textrm{ Übertrag}) \\
1 + 1 = 0 ~(1 \textrm{ Übertrag}) \\
1 = 1 \\
\Rightarrow [10111001]_{2} * [110101]_{2} = [10110010011001]_{2}
\end{align*}
 
 
 
 \subsection*{2.6} 
 Das Komplement einer Zahl $z$ zur Basis $n$ lautet: 
 $$K_b(z) = \begin{cases} b^n− z & \textrm{für } z \neq 0\\
0& \textrm{für } z = 0\end{cases}$$
$$K_{b-1}(z) =  \begin{cases} b^n - b^{-m}− z & \textrm{für } z \neq 0\\
0& \textrm{für } z = 0\end{cases}$$
 Wobei $n$ die Anzahl der Vorkommastellen und $m$ die Anzahl der Nachkommastellen ist.
 \subsubsection*{(a)} Sei $n = 1$:
  
 $K_{10}(4,381)_{10} = 10^1 - 4,381 = 5,619_{10}$
 \subsubsection*{(b)}Sei $n = 0$, $m=4$:
 
 $K_9(0,4172)_{10} = 10^0 - 10^{-4} - 0,4172 = 1 - 0,0001 - 0,4172  = 0,5827_{10}$
 \subsubsection*{(c)}
 Wir rechnen nun im Dualsystem. Sei $n = 1$:
 $$K_2(1,011)_2 = 10^1 - 1,011 = 0,101_2$$
 Rechnung:
 $$
 \begin{array}{r r}
 &10,000\\
 -&\underset{1}{\;}\underset{1}{1},\underset{1}{0}\underset{1}{1}\underset{\;}{1}\\
 \hline
 &00,101 
 \end{array}$$
 \subsubsection*{(d)}Sei $n=100_2$ und $m=10_2$: 
 $$K_1(110,01)_2 = 10^{100} - 10^{-10} - 110,01 = 1000 - 0,01 - 110,01 = 1,1_2$$
 
 Rechnung:
 $$\begin{array}{r r}
 & 1000,00\\
 -&   0,01\\
 -&\underset{1}{\;}\underset{1}{1}\underset{1}{1}\underset{1}{0},\underset{1}{0}1\\
 \hline
 &0001,10
 \end{array}$$
 
 \subsection*{2.7} 
 \begin{center}

 \begin{tabular}{r||r|r|r|r|r}
 Teil & Ganzzahl & Betrag+VZ & Exzess-127 & Einerkomp. & Zweierkomp.\\
 \hline \T
 \textbf{(a) 00001011} &  11 &   11&  116&   11&   11\\
 \textbf{(b) 01100110} & 102 &  102&   15&  102&  102\\
 \textbf{(c) 10000001} & 129 &   -1&   -2& -126& -127\\
 \textbf{(d) 11111110} & 254 & -126& -127&   -1&   -2\\
 \end{tabular}
\end{center}






\end{document}