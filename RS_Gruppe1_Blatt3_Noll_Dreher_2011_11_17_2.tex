\documentclass[12pt,a4paper]{article}
\usepackage[utf8x]{inputenc}
\usepackage{ucs}
\usepackage[german]{babel}
\usepackage{amsmath}
\usepackage{amsfonts}
\usepackage{amssymb}
\usepackage{fancyhdr}
\usepackage{icomma}
\usepackage{booktabs}
\usepackage{bbding}

\pagestyle{fancy}
\fancyhead{}
\fancyfoot{}
\fancyhead[L]{Übung Rechnerstrukturen}
\fancyhead[R]{17. November 2011}
\fancyfoot[R]{Elena Noll, Martin Dreher }
\fancyfoot[C]{\thepage}
\fancyfoot[L]{Matrikelnr: 6335415, 6354762}

\newcommand\T{\rule{0pt}{2.6ex}}
\newcommand\lin{{\tiny \ArrowBoldDownRight}}
\setlength{\parindent}{0pt}

\author{Martin Dreher}
\title{Übung Rechnerstrukturen}

\begin{document}
 

\section*{Aufgabe 3.1}
\subsection*{(a)}
\[5385_{10} - 732_{10}\]
\[= K_(K_5385 + 732)\]
\[= K_(10000-5385 + 732)\]
\[= K_(4615 + 732)\]
\[= K_5347\]
\[= 10000-5347\]
\[= 4653_10\]
\subsection*{(b)}
\[732_{10} - 867_{10}\]
\[= K_(K_732 + 867)\]
\[= K_(1000-732 + 867)\]
\[= K_(268 + 867)\]
\[= K_1135\]
\[= 1000-1135\]
\[= -135_10\]

\section*{Aufgabe 3.2}
\subsection*{(a)}
\[(47,252|3)_{10}\]
\[= 47,252 * 10^3\]
\[= 4,7252 * 10^4\]
\subsection*{(b)} (Komplett mit Basis 2 gerechnet)
\[(-10101,11|-101)_2\]
\[= -10101,11 * 10^{-101}\]
\[= -1010,111 * 10^{-100}\]
\[= -101,0111 * 10^{-11}\]
\[= -10,10111 * 10^{-10}\]
\[= -1,010111 * 10^{-1}\]
\subsection*{(c)}
\[-0,002DA|C)_{16}\]
\[= -0,002DA * 16^C\]
\[= -0,02DA * 16^B\]
\[= -0,2DA * 16^A\]
\[= -2,DA * 16^9\]

\section*{Aufgabe 3.3}
\subsection*{(a)}
\[101 1000\]
\[1011000,0\]
\[Norm:\]
\[1,011000 | 1001\]
\[1001 = 9\]
\[9 + 127 = 136 = 128 + 8 = 10001000\]
\[0 | 10001000 | 01100000000000000000000\]
\subsection*{(b)}
\[-10011011,101\]
\[-1,0011011101 | 1001\]
\[1 | 10001000 | 00110111010000000000000\]

\section*{Aufgabe 3.4}
\[7,516 * 10^6 + 9,9453 * 10^8\]
\[= (0,07516 + 9,9453) 10^8\]
\[...\]

\section*{Aufgabe 3.5}
\[(2,6538 * 10^3) x (3,1415 * 10^5)\]
\[= (0,026538 * 10^5) x (3,1415 * 10^5)\]
\[...\]

\section*{Aufgabe 3.6}
\subsection*{(a)}
Die Übersetzung ist wie folgt:

\begin{tiny}

\texttt{\begin{tabular}{l l l l l l l l l l l l l l l l l l l l l l l l}
0A & 44 & 69 & 65 & 73 & 65 & 0A & 20 & 4C & F6 & 73 & 75 & 6E & 67 & 0A & 20 & 20 & 62 & 72 & 69 & 6E & 67 & 74 & 0A\\
20 & 20 & 20 & 49 & 68 & 6E & 65 & 6E & 0A & 20 & 20 & 20 & 20 & 28 & 66 & 61 & 73 & 74 & 29 & 0A & 20 & 20 & 20 & 20\\
20 & 31 & 35 & 20 & 50 & 75 & 6E & 6B & 74 & 65 & 21 & & & & & & & & & & & & &\\
LF & D & i & e & s & e & LF & SP & L & ö & s & u & n & g & LF & SP & SP & b & r & i & n & g & t & LF\\
SP & SP & SP & I & h & n & e & n & LF & SP & SP & SP & SP & ( & f & a & s & t & ) & LF & SP & SP & SP & SP\\
SP & 15 & P & u & n & k & t & e & ! & & & & & & & & & & & & &
\end{tabular}}
\end{tiny}

Oder als Text:

\texttt{\lin \\
Diese \lin\\
$\cdot$Lösung \lin\\
$\cdot\cdot$bringt \lin\\
$\cdot\cdot\cdot$Ihnen \lin\\
$\cdot\cdot\cdot\cdot$(fast) \lin\\
$\cdot\cdot\cdot\cdot\cdot$15$\cdot$Punkte!}

\subsection*{(b)}
An der Kodierung des Zeilenumbruches als \textit{line feed} ohne \textit{carriage return} erkennt man, dass die Datei Unix-Konform abgespeichert wurde. Daraus allein folgt natürlich noch nichts über die Art des Rechners, der diesen Text erstellt hat. Allerdings haben die Unix-artigen Betriebsysteme, also GNU/Linux, Mac OS X und andere, diese Art des Zeilenumbruches als Standardeinstellung. 
\end{document}